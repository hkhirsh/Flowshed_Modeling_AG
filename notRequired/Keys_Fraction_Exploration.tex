% Options for packages loaded elsewhere
\PassOptionsToPackage{unicode}{hyperref}
\PassOptionsToPackage{hyphens}{url}
%
\documentclass[
]{article}
\usepackage{amsmath,amssymb}
\usepackage{iftex}
\ifPDFTeX
  \usepackage[T1]{fontenc}
  \usepackage[utf8]{inputenc}
  \usepackage{textcomp} % provide euro and other symbols
\else % if luatex or xetex
  \usepackage{unicode-math} % this also loads fontspec
  \defaultfontfeatures{Scale=MatchLowercase}
  \defaultfontfeatures[\rmfamily]{Ligatures=TeX,Scale=1}
\fi
\usepackage{lmodern}
\ifPDFTeX\else
  % xetex/luatex font selection
\fi
% Use upquote if available, for straight quotes in verbatim environments
\IfFileExists{upquote.sty}{\usepackage{upquote}}{}
\IfFileExists{microtype.sty}{% use microtype if available
  \usepackage[]{microtype}
  \UseMicrotypeSet[protrusion]{basicmath} % disable protrusion for tt fonts
}{}
\makeatletter
\@ifundefined{KOMAClassName}{% if non-KOMA class
  \IfFileExists{parskip.sty}{%
    \usepackage{parskip}
  }{% else
    \setlength{\parindent}{0pt}
    \setlength{\parskip}{6pt plus 2pt minus 1pt}}
}{% if KOMA class
  \KOMAoptions{parskip=half}}
\makeatother
\usepackage{xcolor}
\usepackage[margin=1in]{geometry}
\usepackage{color}
\usepackage{fancyvrb}
\newcommand{\VerbBar}{|}
\newcommand{\VERB}{\Verb[commandchars=\\\{\}]}
\DefineVerbatimEnvironment{Highlighting}{Verbatim}{commandchars=\\\{\}}
% Add ',fontsize=\small' for more characters per line
\usepackage{framed}
\definecolor{shadecolor}{RGB}{248,248,248}
\newenvironment{Shaded}{\begin{snugshade}}{\end{snugshade}}
\newcommand{\AlertTok}[1]{\textcolor[rgb]{0.94,0.16,0.16}{#1}}
\newcommand{\AnnotationTok}[1]{\textcolor[rgb]{0.56,0.35,0.01}{\textbf{\textit{#1}}}}
\newcommand{\AttributeTok}[1]{\textcolor[rgb]{0.13,0.29,0.53}{#1}}
\newcommand{\BaseNTok}[1]{\textcolor[rgb]{0.00,0.00,0.81}{#1}}
\newcommand{\BuiltInTok}[1]{#1}
\newcommand{\CharTok}[1]{\textcolor[rgb]{0.31,0.60,0.02}{#1}}
\newcommand{\CommentTok}[1]{\textcolor[rgb]{0.56,0.35,0.01}{\textit{#1}}}
\newcommand{\CommentVarTok}[1]{\textcolor[rgb]{0.56,0.35,0.01}{\textbf{\textit{#1}}}}
\newcommand{\ConstantTok}[1]{\textcolor[rgb]{0.56,0.35,0.01}{#1}}
\newcommand{\ControlFlowTok}[1]{\textcolor[rgb]{0.13,0.29,0.53}{\textbf{#1}}}
\newcommand{\DataTypeTok}[1]{\textcolor[rgb]{0.13,0.29,0.53}{#1}}
\newcommand{\DecValTok}[1]{\textcolor[rgb]{0.00,0.00,0.81}{#1}}
\newcommand{\DocumentationTok}[1]{\textcolor[rgb]{0.56,0.35,0.01}{\textbf{\textit{#1}}}}
\newcommand{\ErrorTok}[1]{\textcolor[rgb]{0.64,0.00,0.00}{\textbf{#1}}}
\newcommand{\ExtensionTok}[1]{#1}
\newcommand{\FloatTok}[1]{\textcolor[rgb]{0.00,0.00,0.81}{#1}}
\newcommand{\FunctionTok}[1]{\textcolor[rgb]{0.13,0.29,0.53}{\textbf{#1}}}
\newcommand{\ImportTok}[1]{#1}
\newcommand{\InformationTok}[1]{\textcolor[rgb]{0.56,0.35,0.01}{\textbf{\textit{#1}}}}
\newcommand{\KeywordTok}[1]{\textcolor[rgb]{0.13,0.29,0.53}{\textbf{#1}}}
\newcommand{\NormalTok}[1]{#1}
\newcommand{\OperatorTok}[1]{\textcolor[rgb]{0.81,0.36,0.00}{\textbf{#1}}}
\newcommand{\OtherTok}[1]{\textcolor[rgb]{0.56,0.35,0.01}{#1}}
\newcommand{\PreprocessorTok}[1]{\textcolor[rgb]{0.56,0.35,0.01}{\textit{#1}}}
\newcommand{\RegionMarkerTok}[1]{#1}
\newcommand{\SpecialCharTok}[1]{\textcolor[rgb]{0.81,0.36,0.00}{\textbf{#1}}}
\newcommand{\SpecialStringTok}[1]{\textcolor[rgb]{0.31,0.60,0.02}{#1}}
\newcommand{\StringTok}[1]{\textcolor[rgb]{0.31,0.60,0.02}{#1}}
\newcommand{\VariableTok}[1]{\textcolor[rgb]{0.00,0.00,0.00}{#1}}
\newcommand{\VerbatimStringTok}[1]{\textcolor[rgb]{0.31,0.60,0.02}{#1}}
\newcommand{\WarningTok}[1]{\textcolor[rgb]{0.56,0.35,0.01}{\textbf{\textit{#1}}}}
\usepackage{graphicx}
\makeatletter
\def\maxwidth{\ifdim\Gin@nat@width>\linewidth\linewidth\else\Gin@nat@width\fi}
\def\maxheight{\ifdim\Gin@nat@height>\textheight\textheight\else\Gin@nat@height\fi}
\makeatother
% Scale images if necessary, so that they will not overflow the page
% margins by default, and it is still possible to overwrite the defaults
% using explicit options in \includegraphics[width, height, ...]{}
\setkeys{Gin}{width=\maxwidth,height=\maxheight,keepaspectratio}
% Set default figure placement to htbp
\makeatletter
\def\fps@figure{htbp}
\makeatother
\setlength{\emergencystretch}{3em} % prevent overfull lines
\providecommand{\tightlist}{%
  \setlength{\itemsep}{0pt}\setlength{\parskip}{0pt}}
\setcounter{secnumdepth}{-\maxdimen} % remove section numbering
\ifLuaTeX
  \usepackage{selnolig}  % disable illegal ligatures
\fi
\IfFileExists{bookmark.sty}{\usepackage{bookmark}}{\usepackage{hyperref}}
\IfFileExists{xurl.sty}{\usepackage{xurl}}{} % add URL line breaks if available
\urlstyle{same}
\hypersetup{
  pdftitle={Keys\_Fraction\_Exploration},
  pdfauthor={Heidi Hirsh},
  hidelinks,
  pdfcreator={LaTeX via pandoc}}

\title{Keys\_Fraction\_Exploration}
\author{Heidi Hirsh}
\date{2024-01-17}

\begin{document}
\maketitle

\begin{Shaded}
\begin{Highlighting}[]
\FunctionTok{rm}\NormalTok{(}\AttributeTok{list=}\FunctionTok{ls}\NormalTok{())}

\DocumentationTok{\#\# Load libraries}
\FunctionTok{library}\NormalTok{(sf)}
\end{Highlighting}
\end{Shaded}

\begin{verbatim}
## Linking to GEOS 3.11.0, GDAL 3.5.3, PROJ 9.1.0; sf_use_s2() is TRUE
\end{verbatim}

\begin{Shaded}
\begin{Highlighting}[]
\FunctionTok{library}\NormalTok{(ggplot2)}
\FunctionTok{library}\NormalTok{(tidyverse)}
\end{Highlighting}
\end{Shaded}

\begin{verbatim}
## -- Attaching core tidyverse packages ------------------------ tidyverse 2.0.0 --
## v dplyr     1.1.3     v readr     2.1.4
## v forcats   1.0.0     v stringr   1.5.0
## v lubridate 1.9.3     v tibble    3.2.1
## v purrr     1.0.2     v tidyr     1.3.0
\end{verbatim}

\begin{verbatim}
## -- Conflicts ------------------------------------------ tidyverse_conflicts() --
## x dplyr::filter() masks stats::filter()
## x dplyr::lag()    masks stats::lag()
## i Use the conflicted package (<http://conflicted.r-lib.org/>) to force all conflicts to become errors
\end{verbatim}

\begin{Shaded}
\begin{Highlighting}[]
\FunctionTok{library}\NormalTok{(viridis)       }
\end{Highlighting}
\end{Shaded}

\begin{verbatim}
## Loading required package: viridisLite
\end{verbatim}

\begin{Shaded}
\begin{Highlighting}[]
\FunctionTok{library}\NormalTok{(wesanderson)}
\FunctionTok{library}\NormalTok{(ggmap)}
\end{Highlighting}
\end{Shaded}

\begin{verbatim}
## The legacy packages maptools, rgdal, and rgeos, underpinning the sp package,
## which was just loaded, were retired in October 2023.
## Please refer to R-spatial evolution reports for details, especially
## https://r-spatial.org/r/2023/05/15/evolution4.html.
## It may be desirable to make the sf package available;
## package maintainers should consider adding sf to Suggests:.
## i Google's Terms of Service: <https://mapsplatform.google.com>
## i Please cite ggmap if you use it! Use `citation("ggmap")` for details.
\end{verbatim}

\begin{Shaded}
\begin{Highlighting}[]
\FunctionTok{library}\NormalTok{(stringr)}
\FunctionTok{library}\NormalTok{(sp)}
\FunctionTok{library}\NormalTok{(raster)}
\end{Highlighting}
\end{Shaded}

\begin{verbatim}
## 
## Attaching package: 'raster'
## 
## The following object is masked from 'package:dplyr':
## 
##     select
\end{verbatim}

\begin{Shaded}
\begin{Highlighting}[]
\FunctionTok{library}\NormalTok{(scales)}
\end{Highlighting}
\end{Shaded}

\begin{verbatim}
## 
## Attaching package: 'scales'
## 
## The following object is masked from 'package:viridis':
## 
##     viridis_pal
## 
## The following object is masked from 'package:purrr':
## 
##     discard
## 
## The following object is masked from 'package:readr':
## 
##     col_factor
\end{verbatim}

\begin{Shaded}
\begin{Highlighting}[]
\FunctionTok{library}\NormalTok{(patchwork)}
\end{Highlighting}
\end{Shaded}

\begin{verbatim}
## 
## Attaching package: 'patchwork'
## 
## The following object is masked from 'package:raster':
## 
##     area
\end{verbatim}

\begin{Shaded}
\begin{Highlighting}[]
\FunctionTok{library}\NormalTok{(mapview)}
\FunctionTok{library}\NormalTok{(leaflet)}
\FunctionTok{library}\NormalTok{(magick)}
\end{Highlighting}
\end{Shaded}

\begin{verbatim}
## Linking to ImageMagick 6.9.12.93
## Enabled features: cairo, fontconfig, freetype, heic, lcms, pango, raw, rsvg, webp
## Disabled features: fftw, ghostscript, x11
\end{verbatim}

\begin{Shaded}
\begin{Highlighting}[]
\FunctionTok{library}\NormalTok{(magrittr)}
\end{Highlighting}
\end{Shaded}

\begin{verbatim}
## 
## Attaching package: 'magrittr'
## 
## The following object is masked from 'package:raster':
## 
##     extract
## 
## The following object is masked from 'package:ggmap':
## 
##     inset
## 
## The following object is masked from 'package:purrr':
## 
##     set_names
## 
## The following object is masked from 'package:tidyr':
## 
##     extract
\end{verbatim}

\begin{Shaded}
\begin{Highlighting}[]
\FunctionTok{library}\NormalTok{(gtools)}

\DocumentationTok{\#\# Functions}
\NormalTok{se }\OtherTok{=} \ControlFlowTok{function}\NormalTok{(x,}\AttributeTok{na.rm=}\NormalTok{T)\{}\FunctionTok{return}\NormalTok{(}\FunctionTok{sd}\NormalTok{(x,}\AttributeTok{na.rm=}\NormalTok{na.rm)}\SpecialCharTok{/}\FunctionTok{sqrt}\NormalTok{(}\FunctionTok{length}\NormalTok{(x)))\}}
\end{Highlighting}
\end{Shaded}

\hypertarget{read-in-carbonate-chemistry-data-via-ana-p-and-proportional-time-calculations-via-thomas-d}{%
\subsection{Read in carbonate chemistry data (via Ana P) and
proportional time calculations (via Thomas
D)}\label{read-in-carbonate-chemistry-data-via-ana-p-and-proportional-time-calculations-via-thomas-d}}

\begin{Shaded}
\begin{Highlighting}[]
\NormalTok{CC }\OtherTok{=} \FunctionTok{read.csv}\NormalTok{(}\StringTok{\textquotesingle{}/Users/heidi.k.hirsh/Desktop/FLK\_data/CCflk\_plusBathy.csv\textquotesingle{}}\NormalTok{) }\CommentTok{\#make sure this is the most appropriate file}
\NormalTok{ptime }\OtherTok{=} \FunctionTok{st\_read}\NormalTok{(}\StringTok{\textquotesingle{}/Users/heidi.k.hirsh/Desktop/FLK\_data/BowtieFractions\_Jan10.shp\textquotesingle{}}\NormalTok{)}
\end{Highlighting}
\end{Shaded}

\begin{verbatim}
## Reading layer `BowtieFractions_Jan10' from data source 
##   `/Users/heidi.k.hirsh/Desktop/FLK_data/BowtieFractions_Jan10.shp' 
##   using driver `ESRI Shapefile'
## Simple feature collection with 38500 features and 5 fields
## Geometry type: POINT
## Dimension:     XY
## Bounding box:  xmin: -81.83393 ymin: 24.39515 xmax: -80.04417 ymax: 25.6475
## Geodetic CRS:  WGS 84
\end{verbatim}

\begin{Shaded}
\begin{Highlighting}[]
\FunctionTok{names}\NormalTok{(ptime)}
\end{Highlighting}
\end{Shaded}

\begin{verbatim}
## [1] "simu"     "sampl_d"  "ndays"    "nrth_fr"  "sth_frc"  "geometry"
\end{verbatim}

\begin{Shaded}
\begin{Highlighting}[]
\DocumentationTok{\#\# Rename (why do names change with the original save?)}
\NormalTok{ptime}\OtherTok{=}\NormalTok{ptime }\SpecialCharTok{\%\textgreater{}\%} \FunctionTok{rename}\NormalTok{(}\AttributeTok{sample\_id=}\NormalTok{sampl\_d,}\AttributeTok{north\_fraction=}\NormalTok{nrth\_fr,}\AttributeTok{south\_fraction=}\NormalTok{sth\_frc) }\CommentTok{\#rename fraction columns}
\FunctionTok{names}\NormalTok{(ptime)}
\end{Highlighting}
\end{Shaded}

\begin{verbatim}
## [1] "simu"           "sample_id"      "ndays"          "north_fraction"
## [5] "south_fraction" "geometry"
\end{verbatim}

\begin{Shaded}
\begin{Highlighting}[]
\DocumentationTok{\#\# Reorder regions and zones to make sense for plotting}
\CommentTok{\# CC$Sub\_region = factor(CC$Sub\_region, levels = c("BB","UK","MK","LK"))}
\NormalTok{CC}\SpecialCharTok{$}\NormalTok{Sub\_region }\OtherTok{=} \FunctionTok{factor}\NormalTok{(CC}\SpecialCharTok{$}\NormalTok{Sub\_region, }\AttributeTok{levels =} \FunctionTok{c}\NormalTok{(}\StringTok{"LK"}\NormalTok{,}\StringTok{"MK"}\NormalTok{,}\StringTok{"UK"}\NormalTok{,}\StringTok{"BB"}\NormalTok{))}
\NormalTok{CC}\SpecialCharTok{$}\NormalTok{Zone }\OtherTok{=} \FunctionTok{factor}\NormalTok{(CC}\SpecialCharTok{$}\NormalTok{Zone, }\AttributeTok{levels =} \FunctionTok{c}\NormalTok{(}\StringTok{"Inshore"}\NormalTok{,}\StringTok{"Mid channel"}\NormalTok{,}\StringTok{"Offshore"}\NormalTok{,}\StringTok{"Oceanic"}\NormalTok{))}
\end{Highlighting}
\end{Shaded}

\hypertarget{assign-each-month-by-number-so-months-dont-plot-misleadingly-in-alphabetical-order-and-years-are-comparable}{%
\subsection{Assign each month by number (so months don't plot
misleadingly in alphabetical order and years are
comparable)}\label{assign-each-month-by-number-so-months-dont-plot-misleadingly-in-alphabetical-order-and-years-are-comparable}}

\begin{Shaded}
\begin{Highlighting}[]
\FunctionTok{head}\NormalTok{(CC}\SpecialCharTok{$}\NormalTok{UTCDate\_Time)}
\end{Highlighting}
\end{Shaded}

\begin{verbatim}
## [1] "2010-03-08 12:26:00" "2010-03-08 13:05:00" "2010-03-08 17:17:00"
## [4] "2010-03-08 17:42:00" "2010-03-08 17:45:00" "2010-03-08 18:22:00"
\end{verbatim}

\begin{Shaded}
\begin{Highlighting}[]
\NormalTok{CC}\SpecialCharTok{$}\NormalTok{date}\OtherTok{=} \FunctionTok{as.Date}\NormalTok{(CC}\SpecialCharTok{$}\NormalTok{Date,}\AttributeTok{format=}\StringTok{"\%Y{-}\%m{-}\%d"}\NormalTok{)}
\NormalTok{CC}\SpecialCharTok{$}\NormalTok{month}\OtherTok{=}\FunctionTok{format}\NormalTok{(CC}\SpecialCharTok{$}\NormalTok{date, }\AttributeTok{format=}\StringTok{"\%B"}\NormalTok{)}
\NormalTok{CC}\SpecialCharTok{$}\NormalTok{month2}\OtherTok{=}\FunctionTok{format}\NormalTok{(CC}\SpecialCharTok{$}\NormalTok{date, }\AttributeTok{format=}\StringTok{"\%m"}\NormalTok{)}
\NormalTok{CC}\SpecialCharTok{$}\NormalTok{month2}\OtherTok{=}\FunctionTok{as.numeric}\NormalTok{(CC}\SpecialCharTok{$}\NormalTok{month2)}
\FunctionTok{head}\NormalTok{(CC}\SpecialCharTok{$}\NormalTok{month2)}
\end{Highlighting}
\end{Shaded}

\begin{verbatim}
## [1] 3 3 3 3 3 3
\end{verbatim}

\end{document}
